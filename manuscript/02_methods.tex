\section{Methods}

\subsection{Experimental Data}

Data were collected from GMR61 \textit{Drosophila} larvae expressing
channelrhodopsins. Animals were tracked at 20 Hz on an agar substrate
while receiving optogenetic LED stimulation in a square-wave pattern: 10
s at peak intensity, 20 s at baseline intensity (30 s cycle). The
present analysis includes 55 tracks containing 1,407 reorientation-onset
events from 2 experimental sessions under the 0-to-250 PWM condition
with constant 7 PWM blue LED background.

Throughout the paper, ``stimulus onset'' refers to the transition from
baseline to peak LED intensity, while ``stimulus offset'' refers to the
transition from peak back to baseline. For the
0-to-250 PWM condition, baseline intensity is 0 (LED off). For the
50-to-250 PWM condition used in factorial analyses, baseline intensity
is 50 PWM (\({\sim}20\%\) of maximum), representing reduced rather than
absent stimulation.

Reorientation events were detected using a curvature-threshold algorithm
that identifies the onset of heading changes. The algorithm captures
large sustained turns and brief head sweeps alike. Events with
measurable duration exceeding 0.1 s were classified as ``true turns'\,'
(\(N=319\)) for behavioral interpretation.

Trajectories were segmented into five behavioral states using rule-based
algorithms adapted from Klein et al.~(2015). Locomotion states include
runs (forward motion with SpeedRunVel \(> 0\)) and pauses (speed below
0.1 mm/s for $\geq 0.5$ s). Reorientation events include turns (heading
changes exceeding 30$^\circ$ within 1 s) and reorientations (turn onset
frames for hazard modeling). Escape behaviors comprise reverse crawls
(backward motion with SpeedRunVel \(< 0\) sustained for $\geq 3$ s). The
five-state segmentation enables extraction of duration statistics for
each behavioral class (Supplementary Figure S7).

Of 701 total tracks across 14 experiments, 79 (11.3\%) spanned the full
20-minute experiment and were used for individual variability analysis.
Incomplete tracks were excluded to avoid right-censoring bias. A
\(t\)-test confirmed no significant difference in mean locomotor speed
between complete and incomplete tracks (\(p = 0.24\)), indicating the
subset is representative. Population-level kernel parameters were
estimated from all tracks using pooled event data.

\subsection{LNP Regression Model Structure}

Reorientation timing is modeled using linear nonlinear Poisson (LNP)
regression, which describes the time-dependent probability that an event
(here, a reorientation) occurs at time \(t\). The instantaneous rate is
given by:

\begin{equation}
\lambda(t) = \exp\left(\beta_0 + u_{\text{track}} + K_{\text{on}}(t_{\text{onset}}) + K_{\text{off}}(t_{\text{offset}})\right)
\label{eq:hazard}
\end{equation}

\noindent where:

\begin{itemize}
    \item $\beta_0 = -6.23$: Calibrated global intercept (log-rate baseline)
    \item $u_{\text{track}} \sim \mathcal{N}(0, \sigma^2)$ with $\sigma = 0.47$: Track-specific random effect capturing individual variability
    \item $K_{\text{on}}(t)$: Stimulus-onset temporal kernel (response to LED intensity increase)
    \item $K_{\text{off}}(t)$: Stimulus-offset temporal kernel (response to LED intensity decrease)
\end{itemize}

The model is fit as a negative-binomial GLM (NB-GLM) with logarithmic
link, treating each video frame (\(dt = 0.05\) s) as a Bernoulli trial
for event occurrence.

\subsection{Stimulus-Onset Kernel: Gamma-Difference}

The stimulus-onset kernel \(K_{\text{on}}(t)\) is parameterized as a
difference of two gamma probability density functions
(Equation \ref{eq:kernel}), where:

\begin{equation}
\Gamma(t; \alpha, \beta) = \frac{t^{\alpha-1} e^{-t/\beta}}{\beta^\alpha \Gamma(\alpha)}
\end{equation}

\noindent is the gamma PDF with shape \(\alpha\) and scale \(\beta\).

\begin{table}[h]
\centering
\caption{Fitted kernel parameters with 95\% bootstrap confidence intervals.}
\label{tab:params}
\begin{tabular}{llll}
\hline
\textbf{Parameter} & \textbf{Value} & \textbf{95\% CI} & \textbf{Interpretation} \\
\hline
$A$ & 0.456 & [0.409, 0.499] & Fast component amplitude \\
$\alpha_1$ & 2.22 & [1.93, 2.65] & Fast shape ($\sim$2 stages) \\
$\beta_1$ & 0.132 s & [0.102, 0.168] & Fast timescale \\
$B$ & 12.54 & [12.43, 12.66] & Slow component amplitude \\
$\alpha_2$ & 4.38 & [4.30, 4.46] & Slow shape ($\sim$4 stages) \\
$\beta_2$ & 0.869 s & [0.852, 0.890] & Slow timescale \\
\hline
\end{tabular}
\end{table}

The fast component peaks at 0.16 s with mean
\(\tau_1 = \alpha_1 \beta_1 = 0.29\) s, while the slow component peaks
at 2.94 s with mean \(\tau_2 = \alpha_2 \beta_2 = 3.81\) s.

\subsection{Stimulus-Offset Kernel}

A separate exponential kernel captures transient effects after LED
intensity decreases:

\begin{equation}
K_{\text{off}}(t) = D \cdot \exp(-t/\tau_{\text{off}})
\label{eq:rebound}
\end{equation}

\noindent with \(D = -0.114\) and \(\tau_{\text{off}} = 2.0\) s. The
negative coefficient represents continued suppression during recovery,
with a half-life of 1.39 s.

\subsection{Event Definition}

The hazard model was fit to all 1,407 inclusive onset events, comprising
salient reorientations (``true turns'\,') as well as minor events such
as head sweeps and curvature fluctuations.

For trajectory simulation and behavioral interpretation, events were
filtered to those with \texttt{turn\_duration} \(> 0.1\) s (\(N = 319\),
23\% of total). Fitting the hazard model to all events while focusing
behavioral output on filtered events follows standard practice in larval
navigation modeling.

For the factorial extension
(Section \ref{sec:factorial}), 12 experiments comprising
7,288 events across 623 tracks were pooled. Two experiments with
anomalously high event counts (approximately \(10\)--\(20\times\) other
sessions) were excluded because their annotation statistics were
inconsistent with the remaining dataset.

\subsection{Rate Calibration}

The NB-GLM intercept (\(\beta_0 = -6.76\)) represents log-hazard per
frame at 20 Hz. Discrete-time simulation with the raw intercept produced
$\sim$60\% of empirical events. A calibration factor was applied:

\begin{equation}
\beta_0^{\text{cal}} = \beta_0 + \log\left(\frac{N_{\text{emp}}}{N_{\text{sim}}}\right) = -6.76 + \log(1.695) = -6.23
\end{equation}

Global rate normalization preserves kernel dynamics (shape and timing)
as well as relative condition effects while matching empirical event
rates. All factorial contrasts are independent of the calibration.

\subsection{Trajectory Simulation}

A run/turn state machine driven by the hazard model was implemented.
During runs, the larva moves forward at 1.0 mm/s with Brownian heading
noise (\(\sigma = 0.03\) rad/\(\sqrt{\text{s}}\)) and transition to a
turn is governed by the instantaneous hazard \(\lambda(t)\). During
turns, heading angle is sampled from
\(\mathcal{N}(\mu = 7^\circ, \sigma = 86^\circ)\) with duration sampled
from a lognormal distribution (median \(= 1.1\) s). Speed is reduced to
\(0.4\times\) the run speed and the larva returns to running after the
turn duration elapses.

The trajectory simulator uses the hazard model to drive run/turn
transitions and reproduces event rates and timing. Spatial statistics
such as path shapes and arena occupancy were not systematically
validated. The simulator demonstrates hazard-driven timing rather than
providing a fully calibrated locomotion model.

\subsection{Validation Metrics}

Model performance was assessed through four metrics grouped by
validation type: fit quality metrics include kernel \(R^2\) and PSTH correlation,
while rate matching metrics include rate ratio and suppression magnitude.

