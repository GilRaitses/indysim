\section{Discussion}

\subsection{Interpretability of the Gamma-Difference Kernel}

The gamma-difference parameterization has a clear biological
interpretation. The shape parameters (\(\alpha_1 \approx 2\),
\(\alpha_2 \approx 4\)) suggest two-stage and four-stage cascades of
first-order processes for the fast and slow components. Shape parameters
of 2 and 4 are consistent with multi-stage signal transduction, where
rapid sensory processing involves photoreceptor activation with 2 stages while slow circuit
adaptation involves synaptic summation with 4 stages.

These timescales match known neurophysiological processes: the fast
timescale (\(\tau_1 = 0.29\) s) corresponds to channelrhodopsin activation
latency and first-order neural responses, while the slow timescale
(\(\tau_2 = 3.81\) s) matches adaptation processes observed in
sensory circuits.

\subsection{Practical Utility}

The analytic kernel supports direct parameter comparison across conditions
and allows testing whether experimental manipulations affect the fast or
slow component. Simulations run directly without requiring precomputed
basis functions.

\begin{figure}[H]
\centering
\includegraphics[width=0.95\textwidth]{data/figures/timescale_variability.png}
\caption{\textbf{Timescale variability across conditions.} (A) Forest plot of fast timescale $\tau_1$ showing 4-fold range from 0.26 s to 1.18 s. (B) Forest plot of slow timescale $\tau_2$ showing relative stability at 3.7--4.5 s. (C) $\tau_1$ vs $\tau_2$ scatter revealing condition clustering. (D) Kernel shapes by condition, with 50-to-250 Constant (green) showing the most delayed peak.}
\label{fig:timescale_variability}
\end{figure}

\subsection{Condition-Dependent Timescales}

The fast timescale \(\tau_1\) shows condition-dependence, ranging from
0.26 s (0-to-250 Cycling) to 1.18 s (50-to-250 Constant). The 4-fold
range in \(\tau_1\) suggests that baseline neural excitation modulates
the speed of sensory transduction. In contrast, the slow timescale
\(\tau_2\) remains relatively stable (3.7--4.5 s) across conditions,
indicating that synaptic adaptation operates on an intrinsic circuit
timescale independent of stimulus parameters.

The 50-to-250 Constant condition, which has persistent low-level LED
activation, produces the slowest fast component. The slowed fast
component may reflect partial adaptation of the rapid sensory pathway
under tonic stimulation, reducing the contrast of stimulus onset and
slowing the initial response.

\subsection{Factorial Design Insights}

The extension to a \(2\times2\) factorial design reveals a dissociation
between stable properties such as kernel shape and timescales in contrast to
condition-dependent properties such as amplitude. Stable kernel shape suggests
that circuit dynamics are intrinsic to the GMR61 pathway, while gain
modulation reflects sensory context.

The partial adaptation effect shows 66\% weaker suppression for 50-to-250,
matching ratio-based scaling as described by the Weber-Fechner law, where
response magnitude depends on the intensity ratio rather than the
absolute level. The cycling background enhancement produces 15\% stronger
suppression and may reflect background-dependent gain modulation; possible
mechanisms include reduced steady-state adaptation or temporal contrast
effects, though the specific circuit basis remains to be determined.

Notably, baseline hazard and LED1-driven suppression gain are
dissociable. Intensity manipulation affects both properties by
reducing baseline turning alongside suppression amplitude, in contrast
to cycling background manipulation which reduces baseline while increasing
suppression gain. The dissociation indicates that tonic excitability
and stimulus-locked modulation are independently tunable circuit
properties.

\subsection{Limitations}

Although point estimates of \(\tau_1\) span approximately 4-fold across
conditions (0.26--1.18 s), formal tests for heterogeneity based on
bootstrap resampling do not reach significance (Cochran's \(Q\)
\(p = 0.252\); \(I^2 = 27\%\), moderate heterogeneity). The observed
range may reflect genuine biological variation or estimator noise given
the sample size. The \(\tau_1\) variation is therefore interpreted as a
hypothesis for future investigation rather than a demonstrated effect.
Effect size analysis (Supplementary Material) reveals that the largest
pairwise difference (0-to-250 Cycling vs 50-to-250 Constant) corresponds
to Hedges' \(g = 3.1\) (large effect), but overlapping confidence
intervals preclude strong claims about condition-specific timescales.

While the factorial model captures main effects well with mean rate ratio =
1.03, the 58\% leave-one-out pass rate with 8/14 experiments within
$\pm$25\% of target does not significantly exceed chance with binomial
\(p = 0.39\). The 58\% pass rate indicates modest rather than strong
out-of-sample validation, likely reflecting substantial
session-to-session variability from experimental factors such as agar
moisture and larval developmental stage not captured by the model.

The primary analysis uses a fixed-effects NB-GLM that pools across
experiments. A robustness check using NB-GLMM with random track
intercepts (Supplementary Table S2b) showed that the kernel parameters
differ by less than 3.5\% between models, with random effect SD
\(= 0.59\) indicating substantial between-track variation in baseline
rate.

Reorientation events were defined using
\texttt{is\_reorientation\_start} from the tracking pipeline, yielding
7,288 events across all factorial conditions. Of these, 77\%
have zero measured duration, representing onset events including micro head sweeps,
while 23\% have duration \(> 0.1\) s and qualify as true turns. The LNP model was
fit to all events to maximize power; trajectory simulations use only the
filtered subset.

Time-rescaling analysis showed modest deviation from the ideal Poisson
model with KS test \(p = 0.17\) and mean deviation 1.3\%, within acceptable
limits for point process models. Remaining deviation may reflect
short-term dependencies such as refractoriness. An explicit post-event
kernel could improve IEI fits but would not change the main LED1-driven
suppression dynamics.

The run/turn simulator omits edge avoidance as well as head sweeps and
speed gradients. The simplified dynamics suffice for demonstrating
hazard-driven timing but limit biomechanical realism.

Using Mason Klein's reverse crawl detection algorithm with SpeedRunVel
\(< 0\) for \(\geq 3\) s, 1,853 reversal events were identified across
all 14 experiments, comprising 2.3\% of total observation time. Reverse
crawls show the \textit{opposite} LED modulation from reorientations:
LED stimulation \textit{increases} reverse crawl probability by 14\%
(\(\chi^2\) test \(p < 0.001\)), with a pronounced spike to 7\% in the
first 0--5 s after LED onset before declining below baseline. The
pattern suggests a biphasic escape response with immediate backward
locomotion followed by suppression of turning. A three-state model
including run, turn, and reverse states could capture the full behavioral repertoire but is
beyond the current scope.

Event detection relies on fixed thresholds of 30\(^\circ\) for turns, 0.5
s for pauses, and 3 s for reverse crawls that have not been formally
sensitivity-tested. Changes in stimulus intensity may alter movement
kinematics such that fixed thresholds misclassify events differently
across conditions, potentially biasing \(\tau_1\) estimates. Threshold
robustness analysis with $\pm$20\% variation and model refitting is
warranted to confirm that the reported timescale differences are not
detection artifacts.

\subsection{Alternative Explanations}

The interpretation that baseline illumination modulates sensory
transduction speed \(\tau_1\) and gain amplitude is presented as a
hypothesis consistent with the data, not a firm conclusion. Several
alternative explanations cannot be excluded:

The \(\tau_1\) variation admits several alternative explanations.
Behavioral state confounds represent one possibility, as larvae in
high-intensity or high-background conditions may differ in arousal,
fatigue, or adaptation states that affect reaction time independently of
sensory transduction. Event detection artifacts offer another
explanation, since fixed turn-angle thresholds applied to conditions
with different movement kinematics could bias which events are detected
and when, shifting apparent \(\tau_1\) without reflecting true sensory
dynamics. Track selection bias cannot be excluded, as the 79 complete
tracks comprising 11\% of total used for variability analysis may differ
systematically from incomplete tracks, though speed comparisons show no
significant difference (\(p = 0.24\)). Fitting noise may also
contribute, given the non-significant heterogeneity test
(\(p = 0.252\)); some of the apparent \(\tau_1\) range may reflect
estimator variance rather than genuine biological differences.

The amplitude variation also admits alternative explanations. Motor
decision probability represents one possibility, as the suppression
amplitude may reflect the probability of initiating a reorientation
given a stimulus as a motor decision rather than purely sensory gain
modulation. State occupancy differences offer another explanation, since
conditions that promote more running versus pausing could shift apparent
amplitude due to different baseline behavioral distributions. Model
mis-specification cannot be excluded, as a gamma-difference kernel that
cannot capture all relevant temporal structure such as
non-stationarities or individual differences may show amplitude
compensation for unmodeled dynamics.

Testing these alternative explanations would require genetic
manipulations to change baseline excitation or hazard-model analyses
that include covariates for speed and run history as well as time since
experiment start.

\subsection{Behavioral Variability and Phenotype Identification}

The event duration distributions (Supplementary Figure S7) reveal
condition-dependent differences in behavioral timing that may serve as
phenotypic signatures. Pause durations are significantly longer in
50-to-250 conditions (\(p < 0.001\), Kruskal-Wallis), suggesting that
baseline illumination affects not only reorientation probability but
also the temporal structure of exploratory pauses. Turn durations also
vary significantly across conditions (\(p < 0.001\)), while run
durations show only marginal differences (\(p = 0.08\)).

Event durations varied significantly for turns and pauses in contrast to
runs which showed only marginal differences, suggesting that
kernel-derived timescales \(\tau_1\) and \(\tau_2\) combined with event
duration statistics could provide a multidimensional behavioral
fingerprint for phenotype identification. Animals with similar
stimulus-response dynamics in kernel shape but different behavioral tempo
in event durations may represent distinct behavioral phenotypes within
the same genotype. Future work should leverage complete tracks spanning
the full experiment to establish individual-level variability bounds
before attempting phenotype clustering.

\subsection{Trial-to-Trial Sensitization}

Analysis of fractional behavior usage across repeated LED pulses
revealed a systematic sensitization effect not captured by the stationary
kernel assumption. Turn fraction increased approximately 2-3 fold over
the 20-minute experiment. At pulse 0, turn fraction ranged from 8-28\%
across conditions; by pulse 17, it reached 37-78\%. All conditions
showed significant positive slopes with \(p < 0.001\) for linear trend.
Cycling background conditions showed faster sensitization with slopes
of \(+0.029\) to \(+0.031\) per pulse compared to constant background
conditions with slopes of \(+0.015\) to \(+0.021\) per pulse. Pause fraction
and reverse crawl fraction remained relatively flat across pulses,
indicating that the sensitization effect is specific to turning behavior
rather than a general arousal change. The increasing turn propensity
over time suggests that animals become more responsive to the LED
stimulus with repeated exposure, possibly through sensitization of the
reorientation circuit. Future models may need to incorporate
trial-to-trial non-stationarity to fully capture behavioral dynamics
across extended experiments.

\subsection{Future Directions}

Several extensions would enhance the model, including temporal
refinements such as a refractory kernel for post-event suppression alongside a
random-effects GLMM for track-level heterogeneity, spatial extensions such as
edge avoidance for bounded arenas alongside chemotaxis integration for gradient
navigation, and downstream applications such as phenotype clustering using kernel parameters and event
duration features.

