\section{Results}

\subsection{Analytic Kernel Captures Temporal Structure}

The 6-parameter gamma-difference kernel closely approximates the
12-parameter raised-cosine reference
(Figure \ref{fig:kernel}A). The analytic form achieves
\(R^2 = 0.968\) and cross-validated \(R^2 = 0.961\) (5-fold,
track-wise), demonstrating that the compact parameterization does not
sacrifice predictive accuracy.


The kernel shows characteristic biphasic dynamics visible in
Figure \ref{fig:kernel}B, with an initial brief increase
in hazard from the fast component with \(\tau_1 = 0.29\) s followed by
sustained suppression from the slow component with \(\tau_2 = 3.81\) s.
The fast component peaks at approximately 0.16 s post-stimulus,
consistent with the latency of channelrhodopsin activation and
first-order neural responses. The slow component dominates from 1--8 s,
producing the characteristic suppression of reorientation probability
during peak-intensity periods. The fast onset followed by sustained
suppression is consistent with rapid sensory transduction giving way to
slower synaptic adaptation.

The stimulus-offset kernel (not shown) shows modest continued
suppression after the return to baseline intensity, with a time constant
of 2.0 s. The 2.0 s time constant indicates that return to baseline
behavior is gradual rather than instantaneous, likely reflecting
recovery from synaptic adaptation.

\subsection{LNP Model Reproduces Event Rates}

Simulation using the calibrated LNP model produces event counts
closely matching empirical observations
(Table \ref{tab:validation};
Figure \ref{fig:validation}).

\begin{table}[h]
\centering
\caption{Validation metrics comparing simulated and empirical data.}
\label{tab:validation}
\begin{tabular}{llll}
\hline
\textbf{Metric} & \textbf{Empirical} & \textbf{Simulated} & \textbf{Status} \\
\hline
Total events & 1,407 & 1,371 & PASS \\
Rate ratio & --- & 0.974 & Target: 0.8--1.25 \\
Baseline rate & $\sim$1.9/min & $\sim$1.9/min & MATCH \\
Peak rate & $\sim$1.0/min & $\sim$1.0/min & MATCH \\
Suppression & $2.0\times$ & $1.9\times$ & MATCH \\
\hline
\end{tabular}
\end{table}

The peri-stimulus time histogram (PSTH) comparison reveals close
agreement between model predictions and empirical observations
(Figure \ref{fig:validation}B). The correlation
between simulated and empirical event histograms is \(r = 0.84\),
indicating good capture of temporal dynamics around stimulus
transitions. The model correctly reproduces the temporal suppression
profile from rapid onset (within 0.5 s of LED activation;
Figure \ref{fig:validation}A, yellow shaded region)
through sustained \({\sim}2\times\) reduction during peak intensity to
gradual recovery following LED offset.

The simulated turn rate comparison
(Figure \ref{fig:validation}C--D) shows strong agreement
between model predictions and empirical observations. The hazard function achieves
$R^2 = 0.962$, confirming that the gamma-difference kernel
captures the underlying temporal dynamics rather than merely matching
aggregate statistics.

\subsection{Trajectory Simulation Matches Behavioral Statistics}

The run/turn simulator driven by the LNP model produces realistic
larval trajectories that recapitulate features of empirical behavior
(Figure \ref{fig:trajectories};
Figure \ref{fig:distributions}). Key behavioral
statistics match empirical observations
(Table \ref{tab:trajectory}).

\begin{table}[h]
\centering
\caption{Trajectory simulation statistics.}
\label{tab:trajectory}
\begin{tabular}{llll}
\hline
\textbf{Metric} & \textbf{Simulated} & \textbf{Empirical} & \textbf{Match} \\
\hline
Turn rate & 1.88/min & 1.84/min & 98\% \\
Mean turn angle & $7^\circ$ & $7^\circ$ & MATCH \\
Turn duration & 1.1 s median & 1.1 s median & MATCH \\
\hline
\end{tabular}
\end{table}

\begin{figure}[H]
\centering
\includegraphics[width=0.95\textwidth]{data/figures/turn_distributions.png}
\caption{\textbf{Empirical turn angle and duration distributions.} Parameters extracted from 319 filtered events with turn duration $> 0.1$ s used to parameterize the trajectory simulator. \textbf{(A)} Turn angle distribution: heading changes show high variability with slight rightward bias. A von Mises distribution shown as a red solid curve with $\kappa = 0.62$ provides the best fit, while a normal distribution shown as a green dashed curve with $\mu = 7^\circ$ and $\sigma = 86^\circ$ provides an adequate alternative fit. The mean absolute turn angle is $69^\circ$, indicating substantial heading changes during reorientation. The small positive bias may reflect asymmetry in the experimental setup or intrinsic larval handedness. \textbf{(B)} Turn duration distribution: event durations follow a lognormal distribution shown as a red curve with shape $s = 0.59$, scale $= 1.29$ s, and median = 1.1 s. Durations range from 0.1--6.8 s, with most turns completing within 2 s. Gamma and exponential fits not shown provided poorer fits to the right tail. The angle and duration distributions enable realistic stochastic simulation of turn kinematics without assuming fixed values.}
\label{fig:distributions}
\end{figure}

Simulated trajectories reproduce the characteristic run-and-turn locomotion pattern observed in larvae (Figure \ref{fig:trajectories}A). During LED on periods, turn suppression extends run segments, matching the empirical observation that optogenetic activation reduces reorientation frequency. The cumulative turn counts in Figure \ref{fig:trajectories}B confirm that simulated and empirical rates converge over the 20-minute observation window despite stochastic variability in individual tracks.

The turn angle distribution in
Figure \ref{fig:distributions}A shows high variability
with \(\sigma = 86^\circ\) and a slight rightward bias
of \(\mu = 7^\circ\), consistent with the exploratory nature of larval
navigation. The turn duration distribution in
Figure \ref{fig:distributions}B follows a lognormal
form with median 1.1 s, matching empirically observed turn durations.
The turn angle and duration distributions were extracted from 319
filtered events with duration \(> 0.1\) s and used to parameterize the
trajectory simulator.

\subsection{Factorial Analysis of Intensity and Background Effects}
\label{sec:factorial}

To assess generalization beyond the reference condition, the model was
extended to a \(2\times2\) factorial design varying LED1 intensity
(0-to-250 vs 50-to-250 PWM) and LED2 background pattern (Constant 7 PWM
vs Cycling 5--15 PWM). The factorial analysis pooled 14 experiments
comprising 701 tracks and 7,867 events across all four conditions
(Figure \ref{fig:factorial}).

The factorial model extends the hazard function with condition-specific
modulation:

\begin{equation}
\log \lambda(t) = \beta_0 + \beta_I \cdot I + \beta_C \cdot C + \beta_{IC} \cdot (I \times C) + (\alpha + \alpha_I \cdot I + \alpha_C \cdot C) \cdot K_{\text{on}}(t) + \gamma \cdot K_{\text{off}}(t)
\label{eq:factorial}
\end{equation}

\noindent where \(I\) indicates the 50-to-250 intensity condition and
\(C\) indicates the cycling background condition.

\begin{table}[ht]
\centering
\caption{Factorial model coefficient estimates with 95\% confidence intervals. All effects are statistically significant at $p < 0.05$.}
\label{tab:factorial}
\small
\begin{tabular}{p{3.2cm}p{1.5cm}p{2.2cm}p{1.2cm}p{3.8cm}}
\hline
\textbf{Effect} & \textbf{Coefficient} & \textbf{95\% CI} & \textbf{Hazard Ratio} & \textbf{Interpretation} \\
\hline
$\beta_I$ (Intensity) & $-0.199$ & $[-0.266, -0.132]$ & $0.82\times$ & Lower baseline hazard \\
$\beta_C$ (Cycling) & $-0.108$ & $[-0.174, -0.042]$ & $0.90\times$ & Lower baseline hazard \\
$\beta_{IC}$ (Interaction) & $-0.119$ & $[-0.218, -0.019]$ & --- & Synergistic reduction \\
$\alpha$ (Kernel amplitude) & $1.005$ & $[0.899, 1.110]$ & --- & Baseline suppression \\
$\alpha_I$ (Intensity mod.) & $-0.665$ & $[-0.773, -0.557]$ & --- & $-66\%$ suppression \\
$\alpha_C$ (Cycling mod.) & $0.152$ & $[0.050, 0.254]$ & --- & $+15\%$ suppression \\
$\gamma$ (Rebound) & $1.669$ & $[0.470, 2.869]$ & --- & Post-offset enhancement \\
\hline
\end{tabular}
\end{table}

\begin{figure}[H]
\centering
\includegraphics[width=0.95\textwidth]{data/figures/figure5_factorial.png}
\caption{\ifxetex\avenirultralight\else\ifluatex\avenirultralight\fi\fi\textbf{Factorial analysis of intensity and background effects.} Extension of the LNP model to a $2\times2$ factorial design with 14 experiments, 701 tracks, and 7,867 events reveals condition-specific modulation of suppression amplitude. \textbf{(A)} Heatmap of suppression amplitude $\alpha + \alpha_I \cdot I + \alpha_C \cdot C$ across the four experimental conditions. Values range from 0.34 for 50-to-250 | Constant with weakest suppression to 1.16 for 0-to-250 | Cycling with strongest suppression, representing a 3.4-fold range. Rows show background pattern with Constant 7 PWM vs Cycling 5--15 PWM; columns show LED1 intensity step with 0-to-250 vs 50-to-250 PWM. Blue indicates weaker suppression; red indicates stronger suppression. \textbf{(B)} Forest plot of factorial coefficient estimates with 95\% confidence intervals. All effects are statistically significant with $p < 0.05$, marked with asterisks. Key findings: (1) Intensity modulation with $\alpha_I = -0.665$ indicates 66\% weaker suppression for the 50-to-250 condition, consistent with partial adaptation to baseline illumination. (2) Cycling background modulation with $\alpha_C = +0.152$ indicates 15\% stronger suppression with oscillating LED2, suggesting reduced adaptation. (3) The baseline effects with $\beta_I = -0.199$ and $\beta_C = -0.108$ show 18\% and 10\% reduction in overall hazard respectively. (4) A synergistic interaction with $\beta_{IC} = -0.119$ indicates greater-than-additive effects when both manipulations are combined. The rebound coefficient $\gamma = 1.67$ is positive, indicating enhanced event probability after LED offset.}
\label{fig:factorial}
\end{figure}

The factorial analysis reveals that both experimental manipulations
significantly modulate the optogenetic response
(Table \ref{tab:factorial};
Figure \ref{fig:factorial}).

The 50-to-250 condition shows 66\% weaker suppression amplitude
with \(\alpha_I = -0.665\) and \(p < 0.001\). The 66\% reduction in
suppression amplitude is consistent with partial adaptation: larvae
pre-exposed to 50 PWM baseline illumination exhibit reduced sensitivity
to the subsequent intensity step, suggesting that the sensory pathway
has already partially adapted to the ambient light level.

The cycling background with LED2 oscillating 5--15 PWM increases
suppression amplitude by 15\% with \(\alpha_C = 0.152\) and \(p = 0.004\). The
modest gain increase may reflect background-dependent modulation of
circuit excitability, possibly through reduced adaptation or temporal
contrast effects. The specific mechanism remains to be determined.

A modest interaction with \(\beta_{IC} = -0.119\) and \(p = 0.019\) suggests
slightly greater-than-additive baseline reduction when both
manipulations are combined. Given limited statistical power for
detecting small interactions (estimated $\sim$30--40\% power for the
observed effect size), the interaction is presented as exploratory and
conclusions are not based on the finding.

The condition-specific suppression amplitudes in
Figure \ref{fig:factorial}A range from 0.34 for 50-to-250
\textbar{} Constant with weakest suppression to 1.16 for 0-to-250 \textbar{} Cycling
with strongest suppression, representing a 3.4-fold range across conditions.

Per-condition kernel fits reveal that the fast timescale \(\tau_1\)
varies 4-fold across conditions (0.26--1.18 s; Figure \ref{fig:timescale_variability}), while the
slow timescale \(\tau_2\) remains stable (3.7--4.5 s). The stability of
\(\tau_2\) alongside variable \(\tau_1\) suggests that baseline
illumination selectively modulates sensory transduction speed without
affecting adaptation dynamics.

Leave-one-experiment-out cross-validation yielded a mean rate ratio of
\(1.03 \pm 0.31\) across 12 held-out experiments (58\% within the
0.8--1.25 target range), indicating reasonable but imperfect
generalization. The substantial inter-experiment variance
(\(\sigma = 0.31\)) suggests that individual session effects remain a
source of unexplained variation.

