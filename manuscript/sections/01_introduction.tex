\section{Introduction}

\subsection{Larval Navigation and Optogenetic Control}

\textit{Drosophila} larvae navigate their environment using a
characteristic locomotor pattern of forward crawling (runs) punctuated
by reorientation maneuvers (turns) during which the animal samples new
heading directions (Gershow et al., 2012; Gomez-Marin et al., 2011). Turn timing and direction are not random but modulated by
sensory input, enabling larvae to perform gradient-based navigation
(climbing and tracking) as well as phototaxis (Gershow et al., 2012;
Kane et al., 2013).

Optogenetic tools provide precise temporal control over neural activity,
allowing researchers to probe how specific circuits influence behavioral
decisions. In GMR61 larvae expressing channelrhodopsins, LED
illumination activates neurons that suppress forward locomotion and
increase the probability of reorientation events (Gepner et al., 2015).
Understanding how turn probability evolves after stimulus onset and
offset is central to modeling sensorimotor integration.

\subsection{LNP Regression and the Role of Temporal Kernels}

Previous work has modeled larval turning probability using linear
nonlinear Poisson (LNP) regression models with temporal basis functions
(Paninski, 2004; Pillow et al., 2008; Hernandez-Nunez et al., 2015;
Klein et al., 2015). LNP regression fits flexible temporal kernels to
capture stimulus-response dynamics. Raised-cosine bases (smooth,
overlapping bump functions centered at different time lags) with many
parameters are common choices for the temporal kernel.

Flexible basis representations give good predictive performance but
offer little interpretability. A 12-parameter raised-cosine kernel may
fit the data well but does not directly reveal the relevant timescales
or separate the effects of sensory transduction and adaptation.

\begin{figure}[ht]
\centering
\includegraphics[width=0.75\textwidth]{data/figures/figure1_kernel.png}
\caption{\textbf{Analytic LNP kernel decomposition.} \textbf{(A)} Full kernel K$_{\text{on}}$(t) as a blue solid line; red shading marks suppression where K$_{\text{on}}$(t) $< 0$. Peak suppression of $-3.06$ log-hazard units at $t^* = 2.9$ s corresponds to turn rate dropping to $\sim$5\% of baseline. \textbf{(B)} Component decomposition: fast excitatory component (green; $\tau_1 = 0.29$ s) captures rapid transduction; slow suppressive component (red; $\tau_2 = 3.81$ s) represents adaptation. Combined kernel (blue dashed) matches Panel A.}
\label{fig:kernel}
\end{figure}

\subsection{Contribution}

An analytic temporal kernel addresses the interpretability gap for
LNP regression models of optogenetically-driven reorientation. The
kernel is a difference of two gamma probability density functions:

\begin{equation}
K_{\text{on}}(t) = A \cdot \Gamma(t; \alpha_1, \beta_1) - B \cdot \Gamma(t; \alpha_2, \beta_2)
\label{eq:kernel}
\end{equation}

\noindent where \(\Gamma(t; \alpha, \beta)\) denotes the gamma
probability density function with shape \(\alpha\) and scale \(\beta\).

The gamma-difference kernel captures a fast excitatory component peaking
at 0.16 s with mean \(\tau_1 = 0.29\) s, representing rapid sensory
transduction, alongside a slow suppressive component peaking at 2.9 s with mean
\(\tau_2 = 3.81\) s, representing synaptic or network adaptation.

The gamma-difference form arises naturally as the impulse response of
cascaded first-order processes, making parameters directly interpretable
in terms of processing stages and time constants.

Validation against GMR61 data under 10 s peak / 20 s baseline LED stimulation
supports kernel accuracy. The analytic form matches both the 12-basis
reference with \(R^2 = 0.968\) and empirical event rates with ratio = 0.97.
The kernel also drives a trajectory simulator that recapitulates
observed behavior.

