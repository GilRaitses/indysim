\section{Conclusions}

An analytic LNP kernel for optogenetically-driven larval
reorientation is presented that combines interpretability with
predictive accuracy. The 6-parameter gamma-difference form captures two
biologically meaningful timescales \(\tau_1 = 0.29\) s and
\(\tau_2 = 3.81\) s and reproduces empirical event statistics with high
fidelity with rate ratio \(= 0.97\) and \(R^2 = 0.968\).

Extension to a \(2\times2\) factorial design with 14 experiments, 701
tracks, and 7,867 events reveals conserved kernel shape alongside
condition-dependent amplitude, with baseline intensity effects showing 66\%
suppression reduction for 50-to-250 PWM in contrast to background
pattern effects showing 15\% suppression increase for cycling. All factorial
effects are statistically significant, indicating that the model can
quantify condition-specific modulation of sensorimotor processing.

Embedded in a run/turn trajectory simulator, the model generates
realistic larval behavior that matches observed turn rates of 1.88 vs 1.84
turns/min. The LNP model framework provides a starting point for
quantitative analysis of sensorimotor processing across experimental
conditions and genotypes.

