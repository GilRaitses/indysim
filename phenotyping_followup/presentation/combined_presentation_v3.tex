\documentclass[aspectratio=169,11pt]{beamer}
\usetheme{metropolis}
\usepackage{graphicx}
\usepackage{amsmath}
\usepackage{booktabs}

% Reduce bottom margin for captions
\setbeamertemplate{frametitle}[default][left]
\addtobeamertemplate{frametitle}{}{\vspace{-0.3em}}

\title{Sensorimotor Habituation in \textit{Drosophila} Larvae}
\subtitle{Population-Level Modeling and Individual Phenotyping Validation}
\author{Gil Raitses}
\institute{Syracuse University}
\date{December 22, 2025}

\begin{document}

\begin{frame}
\titlepage
\end{frame}

%% ===========================================
%% PART 1: ORIGINAL STUDY
%% ===========================================

\begin{frame}{Executive Summary -- Original Study}
\textbf{Population-Level Sensorimotor Habituation Model}

\vspace{0.3cm}
Larval reorientation behavior follows a gamma-difference kernel with two distinct timescales that capture the dynamics of sensory response and adaptation.

\vspace{0.3cm}
The fast excitatory component with $\tau_1 \approx 0.3$ seconds drives the initial response to optogenetic light onset. The slow inhibitory component with $\tau_2 \approx 4$ seconds produces delayed suppression that underlies behavioral habituation across repeated stimuli.

\vspace{0.3cm}
The model was validated across 14 experiments comprising 701 unique larval tracks collected under four stimulation conditions.

\vspace{0.4cm}
\textbf{Key Result} \quad The gamma-difference kernel accurately predicts population-level reorientation dynamics and provides biologically interpretable timescales.
\end{frame}

\begin{frame}{Kernel Structure}
\centering
\includegraphics[height=0.68\textheight]{../../figures/figure1_kernel.png}

\vspace{0.15cm}
\small The gamma-difference kernel $K(t) = A \cdot \Gamma(t;\alpha_1,\beta_1) - B \cdot \Gamma(t;\alpha_2,\beta_2)$ modulates reorientation hazard rate following LED onset. The left panel shows the combined kernel with peak excitation at 0.3 seconds followed by suppression reaching minimum at 3 seconds. The right panel decomposes this into fast excitatory gamma in green with $\tau_1 = 0.29$s and slow suppressive gamma in red with $\tau_2 = 3.81$s.
\end{frame}

\begin{frame}{Simulated vs Empirical Event Counts}
\centering
\includegraphics[height=0.68\textheight]{../figures/fig_simulation_design.pdf}

\vspace{0.15cm}
\small Validation of the simulation model against empirical data. Panel A shows overlapping histograms of event counts per track for 260 empirical tracks in cyan and 300 simulated tracks in pink. Panel B shows box plots comparing the distributions. The simulated tracks were generated using the fitted population kernel and match the empirical distribution with similar median of 15 events and interquartile range.
\end{frame}

\begin{frame}{Habituation Dynamics}
\centering
\includegraphics[height=0.65\textheight]{../../figures/figure_habituation.png}

\vspace{0.15cm}
\small Fraction of time spent turning increases across successive LED pulses in all four experimental conditions. Orange and green lines show 0-250 PWM conditions while pink and purple show 50-250 PWM conditions. Steeper slopes indicate faster habituation. The 0-250 Cycling condition shows the strongest habituation effect with slope +0.031 per pulse. Shaded bands indicate 95\% confidence intervals. Asterisks mark pulses significantly different from pulse 0.
\end{frame}

\begin{frame}{Behavioral State Analysis}
\centering
\includegraphics[height=0.65\textheight]{figures/figure_behavior_stacked_cinnamoroll.png}

\vspace{0.15cm}
\small Fractional behavioral state usage across LED pulses for all four experimental conditions. Gray indicates forward running. Pink indicates turning behavior. Blue indicates pausing. Orange indicates reverse crawling. Turning fraction increases dramatically across pulses in all conditions while running fraction decreases. Pausing remains below 5\% throughout. The 50-250 Cycling condition shows the largest shift from running to turning behavior by pulse 17.
\end{frame}

\begin{frame}{Leave-One-Experiment-Out Validation}
\centering
\includegraphics[height=0.65\textheight]{../../figures/loeo_permutation_null.png}

\vspace{0.15cm}
\small LOEO cross-validation assesses whether kernel parameters generalize across experiments. The histogram shows the null distribution generated from 1000 permutations of experiment labels. The observed pass rate of 50\% falls within the null distribution with p=0.618 indicating that cross-experiment generalization is no better than chance. Model performs reliably at the population level but individual experiments show high variability.
\end{frame}

%% ===========================================
%% PART 2: FOLLOW-UP STUDY
%% ===========================================

\begin{frame}{Executive Summary -- Follow-Up Study}
\textbf{Individual-Level Phenotyping Validation}

\vspace{0.3cm}
The follow-up study asked whether individual larvae can be phenotyped using kernel parameters. The answer is negative with current protocols.

\vspace{0.3cm}
Challenge: Sparse data with only 18 to 25 events per 10 to 20 minute track makes 6-parameter kernel estimation unreliable.

\vspace{0.3cm}
Finding: Apparent phenotypic clusters identified by K-means are statistical artifacts of sparse data. Gap statistic optimization suggests k=1 is optimal indicating no discrete phenotypes exist.

\vspace{0.3cm}
Only 8.6\% of tracks show genuine individual differences that exceed measurement noise.

\vspace{0.3cm}
\textbf{Key Result} \quad Individual-level phenotyping requires experimental protocol modifications before reliable kernel-based phenotyping becomes feasible.
\end{frame}

%% Clustering Illusion - Use ORIGINAL multi-panel figure (Dec 21 12:52)
\begin{frame}{The Clustering Illusion}
\centering
\includegraphics[height=0.72\textheight]{../figures/core/fig1_clustering_illusion.pdf}

\vspace{0.1cm}
\small Panel A shows true PCA distribution revealing unimodal structure with some outliers. Panel B shows all four validation methods failed with ARI below 0.13. Panel C shows gap statistic is minimized at k=1 indicating no discrete clusters exist.
\end{frame}

%% Data Sparsity - Use ORIGINAL multi-panel figure (Dec 21 12:52)
\begin{frame}{Data Sparsity Explains Instability}
\centering
\includegraphics[height=0.72\textheight]{../figures/core/fig2_data_sparsity.pdf}

\vspace{0.1cm}
\small Panel A shows mean 25 events per track versus the 100 required for stable estimation. Panel B shows many MLE estimates fall in the implausible range above 1.5 seconds. Panel C shows the math problem: 4 parameters divided by 25 events equals a 6:1 ratio which is underdetermined.
\end{frame}

%% Hierarchical Shrinkage - Use ORIGINAL multi-panel figure (Dec 21 12:52)
\begin{frame}{Hierarchical Shrinkage}
\centering
\includegraphics[height=0.72\textheight]{../figures/core/fig3_hierarchical_shrinkage.pdf}

\vspace{0.1cm}
\small Bayesian estimates shrink toward population mean proportionally to data sparsity. Tracks with extreme MLE values show the most shrinkage while tracks with abundant data retain individual estimates.
\end{frame}

%% Identifiability - Use ORIGINAL multi-panel figure (Dec 21 12:52)
\begin{frame}{The Identifiability Problem}
\centering
\includegraphics[height=0.72\textheight]{../figures/fig2_identifiability_v3.pdf}

\vspace{0.1cm}
\small Panel A shows continuous design produces high bias and RMSE. Panel B shows burst design extracts 10x more Fisher Information. Panel C shows MLE recovery differs by design. Panel D explains why continuous fails: inhibition dominates.
\end{frame}

%% Stimulation Protocols - Use ORIGINAL multi-panel figure (Dec 21 12:52)
\begin{frame}{Stimulation Protocol Comparison}
\centering
\includegraphics[height=0.72\textheight]{../figures/fig_stimulation_schematic.pdf}

\vspace{0.1cm}
\small Four LED stimulation designs showing the 30-second cycle structure. Panel A shows current continuous 10s design with low power. Panel B shows recommended burst 10x0.5s with 8x more information. Panels C and D show intermediate alternatives.
\end{frame}

%% Kernel Comparison - Use ORIGINAL figure (Dec 21 12:52)
\begin{frame}{Kernel Model Comparison}
\centering
\includegraphics[height=0.65\textheight]{../figures/fig7_model_comparison.pdf}

\vspace{0.15cm}
\small Model comparison shows the gamma-difference kernel achieves R$^2$ = 0.968 with 6 parameters. The raised cosine basis achieves R$^2$ = 0.974 but requires 12 parameters. The gamma-difference form provides biological interpretability with equivalent fit quality.
\end{frame}

%% ===========================================
%% RECOMMENDATIONS - 5 SLIDES WITH FIGURES
%% ===========================================

\begin{frame}{Recommendation 1 -- Protocol Modification}
\centering
\includegraphics[height=0.65\textheight]{../figures/fig4_recommendations_v2.pdf}

\vspace{0.15cm}
\small Replace continuous 10-second ON periods with burst trains. Burst design samples the early excitatory window where $\tau_1$ is identifiable. Each burst event carries 10 times more Fisher information than continuous events. This modification alone could reduce the number of events required for reliable estimation from 100 to 30.
\end{frame}

\begin{frame}{Recommendation 2 -- Extended Recording}
\centering
\includegraphics[height=0.65\textheight]{../figures/fig5_power_analysis.pdf}

\vspace{0.15cm}
\small Target 40 minutes or more of recording to achieve at least 50 reorientation events per track. Current 10 to 20 minute recordings yield only 18 to 25 events which is insufficient for 6-parameter estimation. Power analysis indicates that 100 events are required for 80\% power to detect a 0.2 second difference in $\tau_1$ at the individual level.
\end{frame}

\begin{frame}{Recommendation 3 -- Model Simplification}
\centering
\includegraphics[height=0.65\textheight]{../figures/core/fig3_hierarchical_shrinkage.pdf}

\vspace{0.15cm}
\small Reduce the parameter space by fixing population-derived parameters. Fix $\tau_2$ at the population estimate of 3.8 seconds and fix the amplitude ratio B/A at the population value of 8. Estimate only the fast timescale $\tau_1$ per individual track. Hierarchical Bayesian estimation provides natural regularization toward the population mean.
\end{frame}

\begin{frame}{Recommendation 4 -- Alternative Phenotypes}
\centering
\includegraphics[height=0.65\textheight]{../figures/core/fig4_fast_responders.pdf}

\vspace{0.15cm}
\small Use robust composite phenotypes that avoid kernel fitting entirely. The ON/OFF event ratio measures whether larvae respond preferentially during LED-ON versus LED-OFF periods and requires only event counts not full kernel estimation. First-event latency measures the time from LED onset to first reorientation and captures response speed directly.
\end{frame}

\begin{frame}{Recommendation 5 -- Within-Condition Analysis}
\centering
\includegraphics[height=0.65\textheight]{../figures/core/fig1_clustering_illusion.pdf}

\vspace{0.15cm}
\small Analyze individual differences within experimental conditions rather than pooling across conditions. When data from different stimulation intensities and temporal patterns are pooled the condition effects dominate and mask genuine individual variation. The ARI near zero across all validation methods indicates no reproducible structure when pooling.
\end{frame}

%% ===========================================
%% CONCLUSIONS
%% ===========================================

\begin{frame}{Conclusions -- Original Study}
\textbf{Population-Level Modeling Success}

\vspace{0.3cm}
The gamma-difference kernel accurately models population-level reorientation dynamics across 14 experiments and 701 unique larval tracks.

\vspace{0.3cm}
Two timescales govern behavioral dynamics. Fast excitation with $\tau_1 \approx 0.3$ seconds captures the initial sensory response to optogenetic stimulation. Slow suppression with $\tau_2 \approx 4$ seconds captures habituation-like adaptation that accumulates across repeated stimuli.

\vspace{0.3cm}
The model is robust across experimental conditions as demonstrated by leave-one-experiment-out cross-validation and bootstrap confidence intervals.

\vspace{0.3cm}
The gamma-difference form provides biological interpretability that flexible basis function models lack while achieving equivalent goodness of fit.
\end{frame}

\begin{frame}{Conclusions -- Follow-Up Study}
\textbf{Individual Phenotyping Challenges}

\vspace{0.3cm}
Individual phenotyping using kernel parameters fails with current experimental protocols due to fundamentally sparse data.

\vspace{0.3cm}
Apparent clusters identified by K-means or hierarchical clustering are statistical artifacts of fitting high-dimensional models to low-event tracks rather than genuine discrete phenotypes.

\vspace{0.3cm}
Only 8.6\% of tracks show individual variation that exceeds the measurement noise floor.

\vspace{0.3cm}
Current protocols achieve only 20 to 30\% statistical power for detecting fast responder phenotypes.

\vspace{0.4cm}
\textbf{Bottom Line} \quad Population-level analysis is robust and biologically meaningful. Individual phenotyping requires experimental redesign before kernel-based classification becomes reliable.
\end{frame}

\begin{frame}
\centering
\Huge Thank You

\vspace{1cm}
\Large Questions?
\end{frame}

%% ===========================================
%% FAQ SECTION
%% ===========================================

\begin{frame}{FAQ -- Original Study Methods}
\textbf{What is the sequence of processes in the original study?}

\vspace{0.3cm}
\textbf{Data collection} \quad 14 optogenetic experiments with 701 larval tracks under four stimulation conditions varying intensity from 0-250 PWM versus 50-250 PWM and temporal pattern from constant versus cycling.

\vspace{0.2cm}
\textbf{Trajectory extraction} \quad MAGAT Analyzer extracts larval trajectories from video and segments behavioral states including runs and reorientations.

\vspace{0.2cm}
\textbf{Event detection} \quad Reorientation onset times are extracted from MAGAT segmentation as point events for hazard modeling.

\vspace{0.2cm}
\textbf{Kernel fitting} \quad Population-level gamma-difference kernel is fitted using maximum likelihood estimation on the pooled PSTH.

\vspace{0.2cm}
\textbf{Validation} \quad Leave-one-experiment-out cross-validation and bootstrap confidence intervals assess generalization and parameter uncertainty.
\end{frame}

\begin{frame}{FAQ -- Follow-Up Study Methods}
\textbf{What processes were used in the follow-up study?}

\vspace{0.3cm}
\textbf{Individual fitting} \quad Maximum likelihood estimation of 6-parameter gamma-difference kernel per track using the same event data.

\vspace{0.2cm}
\textbf{Clustering} \quad K-means and hierarchical clustering applied to individual kernel parameters to identify putative phenotypic groups.

\vspace{0.2cm}
\textbf{Validation} \quad Round-trip validation simulates from fitted kernels and re-clusters to test reproducibility. Gap statistic tests whether k clusters improve over k=1.

\vspace{0.2cm}
\textbf{Power analysis} \quad Simulation-based analysis estimates statistical power for phenotype detection as a function of event count and protocol design.

\vspace{0.2cm}
\textbf{Identifiability analysis} \quad Fisher Information quantifies how much information each event contains about kernel parameters under different stimulation designs.
\end{frame}

\begin{frame}{FAQ -- Why does population modeling succeed but individual fails?}
\textbf{The key is the data-to-parameter ratio}

\vspace{0.3cm}
Population modeling pools approximately 15,000 reorientation events across 701 tracks to estimate 6 kernel parameters. The data-to-parameter ratio exceeds 2,500 to 1 providing strong statistical power.

\vspace{0.3cm}
Individual modeling attempts to estimate the same 6 parameters from only 18 to 25 events per track. The data-to-parameter ratio is only 3 to 5 which is far below the minimum of 10 to 1 needed for reliable maximum likelihood estimation.

\vspace{0.3cm}
The gamma-difference kernel is overparameterized for individual-level sparse data. Identifiability analysis reveals that only $\tau_1$ and possibly the amplitude ratio can be reliably estimated with fewer than 50 events.

\vspace{0.3cm}
Hierarchical Bayesian estimation provides partial mitigation by shrinking individual estimates toward the population mean but cannot create information that is not present in the data.
\end{frame}

\begin{frame}{FAQ -- What is hierarchical shrinkage?}
\textbf{Regularization through shared population structure}

\vspace{0.3cm}
Hierarchical Bayesian models assume individual parameters are drawn from a population distribution with unknown mean and variance.

\vspace{0.3cm}
Tracks with sparse data are shrunk toward the population mean because their individual estimates are uncertain. Tracks with abundant data retain individual estimates because they contain sufficient information.

\vspace{0.3cm}
The amount of shrinkage is determined automatically by the posterior distribution. More sparse data produces more shrinkage.

\vspace{0.3cm}
Shrinkage is not a bias correction. It is optimal regularization under the assumption that individuals are exchangeable members of a population.

\vspace{0.3cm}
In the data sparsity regime of this study almost all individual estimates are shrunk heavily toward the population mean confirming that individual phenotyping is not feasible without protocol changes.
\end{frame}

\begin{frame}{FAQ -- How should clustering results be interpreted?}
\textbf{Clusters in sparse high-dimensional data are usually artifacts}

\vspace{0.3cm}
K-means and hierarchical clustering will always produce k clusters regardless of whether true clusters exist in the data.

\vspace{0.3cm}
The gap statistic compares within-cluster dispersion to a null reference and finds that k=1 is optimal meaning no discrete clusters are present.

\vspace{0.3cm}
Round-trip validation tests whether clusters are reproducible by simulating new data from fitted parameters and re-clustering. Agreement below ARI = 0.2 indicates clusters are not reproducible.

\vspace{0.3cm}
The continuous unimodal PCA distribution confirms that kernel parameter variation is smooth rather than discrete.

\vspace{0.3cm}
Future clustering should be attempted only after protocol modifications increase event counts to 100 or more per track and reduce the parameter space to 1 or 2 dimensions.
\end{frame}

\end{document}

