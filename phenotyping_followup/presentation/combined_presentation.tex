\documentclass[aspectratio=169,11pt]{beamer}
\usetheme{metropolis}
\usepackage{graphicx}
\usepackage{amsmath}
\usepackage{booktabs}

% Colors
\definecolor{darkblue}{RGB}{0,51,102}
\definecolor{lightgray}{RGB}{245,245,245}

\title{Sensorimotor Habituation in \textit{Drosophila} Larvae}
\subtitle{Population-Level Modeling and Individual Phenotyping Validation}
\author{Gil Raitses}
\institute{Syracuse University}
\date{December 22, 2025}

\begin{document}

\begin{frame}
\titlepage
\end{frame}

%% ===========================================
%% PART 1: ORIGINAL STUDY
%% ===========================================

\begin{frame}{Executive Summary -- Original Study}
\textbf{Population-Level Sensorimotor Habituation Model}

\begin{itemize}
    \item Larval reorientation behavior follows a \textbf{gamma-difference kernel} with two timescales
    \item Fast excitatory component ($\tau_1 \approx 0.3$s) drives initial response
    \item Slow inhibitory component ($\tau_2 \approx 4$s) produces suppression
    \item Model validated across 14 experiments with 701 tracks
    \item Leave-one-experiment-out cross-validation confirms robustness
\end{itemize}

\vspace{0.5cm}
\textbf{Key Result} \quad The gamma-difference kernel accurately predicts population-level reorientation dynamics under optogenetic stimulation.
\end{frame}

\begin{frame}{Kernel Structure}
\centering
\includegraphics[height=0.8\textheight]{../../figures/figure1_kernel.png}

The gamma-difference kernel $K(t) = A \cdot \Gamma(t; \alpha_1, \beta_1) - B \cdot \Gamma(t; \alpha_2, \beta_2)$ modulates reorientation hazard rate. Fast excitation peaks at $\sim$0.3s; slow suppression persists for $\sim$4s.
\end{frame}

\begin{frame}{Model Validation}
\centering
\includegraphics[height=0.8\textheight]{../../figures/figure2_validation.png}

Cross-validation demonstrates model robustness. Fitted kernels generalize across experiments with consistent $\tau_1$ and $\tau_2$ estimates.
\end{frame}

\begin{frame}{Trajectory Analysis}
\centering
\includegraphics[height=0.8\textheight]{../../figures/figure3_trajectories.png}

Example larval trajectories showing reorientation events aligned to LED stimulation cycles. The kernel captures event clustering after stimulus onset.
\end{frame}

\begin{frame}{Habituation Dynamics}
\centering
\includegraphics[height=0.8\textheight]{../../figures/figure_habituation.png}

Habituation effects across repeated stimulation cycles. Response magnitude decreases with cumulative exposure, consistent with sensorimotor adaptation.
\end{frame}

\begin{frame}{Factorial Design}
\centering
\includegraphics[height=0.8\textheight]{../../figures/figure5_factorial.png}

Factorial analysis of kernel parameters across experimental conditions. The fast timescale $\tau_1$ varies 4-fold across baseline illumination levels.
\end{frame}

\begin{frame}{Behavioral State Analysis}
\centering
\includegraphics[height=0.8\textheight]{../../figures/figure_behavior_stacked_cinnamoroll.png}

Fractional time in behavioral states including run, turn, and head swing across stimulation protocols. LED-ON periods show increased turn fraction.
\end{frame}

\begin{frame}{Leave-One-Experiment-Out Validation}
\centering
\includegraphics[height=0.8\textheight]{../../figures/loeo_permutation_null.png}

LOEO permutation test. Observed log-likelihood ratio exceeds 95\% of null distribution, confirming kernel generalization.
\end{frame}

%% ===========================================
%% PART 2: FOLLOW-UP STUDY
%% ===========================================

\begin{frame}{Executive Summary -- Follow-Up Study}
\textbf{Individual-Level Phenotyping Validation}

\begin{itemize}
    \item \textbf{Question} \quad Can individual larvae be phenotyped using kernel parameters?
    \item \textbf{Challenge} \quad Sparse data with $\sim$18--25 events per 10--20 min track
    \item \textbf{Finding} \quad Apparent phenotypic clusters are artifacts of sparse data
    \item Gap statistic suggests optimal $k=1$ cluster indicating no discrete phenotypes
    \item Round-trip validation ARI = 0.128 falls below 0.5 threshold
    \item Only 8.6\% of tracks show genuine individual differences
\end{itemize}

\vspace{0.5cm}
\textbf{Key Result} \quad Population-level analysis is robust; individual-level phenotyping requires protocol modifications including burst stimulation and longer recordings.
\end{frame}

\begin{frame}{The Clustering Illusion}
\centering
\includegraphics[height=0.8\textheight]{../figures/core/fig1_clustering_illusion.pdf}

PCA reveals unimodal distribution rather than discrete clusters. All validation methods failed with ARI $<$ 0.5. Gap statistic suggests optimal $k=1$.
\end{frame}

\begin{frame}{Data Sparsity Challenge}
\centering
\includegraphics[height=0.8\textheight]{../figures/core/fig2_data_sparsity.pdf}

With only $\sim$18 events per track and 6 kernel parameters, the data-to-parameter ratio is 3 to 1. Recommended ratio is 10 to 1.
\end{frame}

\begin{frame}{Hierarchical Shrinkage}
\centering
\includegraphics[height=0.8\textheight]{../figures/core/fig3_hierarchical_shrinkage.pdf}

Hierarchical Bayesian model shrinks extreme MLE estimates toward population mean of $\tau_1 = 0.63$s. Only 8.6\% are genuine outliers.
\end{frame}

\begin{frame}{Candidate Fast Responders}
\centering
\includegraphics[height=0.8\textheight]{../figures/core/fig4_fast_responders.pdf}

22 candidate fast-responder tracks representing 8.6\% show $\tau_1 \approx 0.45$s versus population mean of 0.63s. Independent validation required.
\end{frame}

\begin{frame}{Power Analysis}
\centering
\includegraphics[height=0.8\textheight]{../figures/fig5_power_analysis.pdf}

Current data achieves only 20--30\% power to detect $\Delta\tau_1 = 0.2$s. Reaching 80\% power requires $\sim$100 events per track.
\end{frame}

\begin{frame}{Identifiability Problem}
\centering
\includegraphics[height=0.8\textheight]{../figures/fig2_identifiability_v3.pdf}

Fisher Information analysis reveals burst stimulation provides 10$\times$ higher information for $\tau_1$ than continuous stimulation.
\end{frame}

\begin{frame}{Design Comparison}
\centering
\includegraphics[height=0.8\textheight]{../figures/fig_design_comparison_summary.pdf}

Optimal design depends on kernel regime. For inhibition-dominated kernels in current data, burst stimulation is required.
\end{frame}

\begin{frame}{Stimulation Protocol Recommendations}
\centering
\includegraphics[height=0.8\textheight]{../figures/fig_stimulation_schematic.pdf}

Recommended burst design uses 10 pulses of 0.5s ON with 0.5s gaps. This achieves 8$\times$ more informative events than continuous 10s ON.
\end{frame}

\begin{frame}{Kernel Model Comparison}
\centering
\includegraphics[height=0.8\textheight]{../figures/fig_kernel_comparison.pdf}

Gamma-difference kernel with 6 parameters achieves R$^2$ = 0.968 compared to raised cosine basis with 12 parameters, validating the parametric form.
\end{frame}

%% ===========================================
%% CONCLUSIONS
%% ===========================================

\begin{frame}{Conclusions}
\textbf{Original Study}
\begin{itemize}
    \item Gamma-difference kernel accurately models population-level reorientation dynamics
    \item Two timescales govern behavior including fast excitation $\tau_1$ and slow suppression $\tau_2$
    \item Robust across 14 experiments via LOEO cross-validation
\end{itemize}

\vspace{0.5cm}
\textbf{Follow-Up Study}
\begin{itemize}
    \item Individual phenotyping fails with current protocols due to sparse data
    \item Apparent clusters are artifacts rather than genuine phenotypes
    \item 8.6\% candidate fast responders require independent validation
    \item Recommendations include burst stimulation, $\geq$100 events per track, and composite phenotypes
\end{itemize}
\end{frame}

\begin{frame}{Recommendations for Future Work}
\begin{enumerate}
    \item \textbf{Protocol modification} \quad Replace 10s continuous ON with burst trains of 10 $\times$ 0.5s pulses
    \item \textbf{Extended recording} \quad Target 40+ minutes to achieve $\geq$50 events per track
    \item \textbf{Model simplification} \quad Fix $\tau_2$, $A$, $B$ at population values and estimate only $\tau_1$
    \item \textbf{Alternative phenotypes} \quad Use ON/OFF ratio and first-event latency which are robust with sparse data
    \item \textbf{Within-condition analysis} \quad Avoid confounding by condition effects
\end{enumerate}

\vspace{0.5cm}
\textbf{Bottom line} \quad Population-level analysis is robust and publishable. Individual phenotyping requires experimental redesign.
\end{frame}

\begin{frame}
\centering
\Huge Thank You

\vspace{1cm}
\Large Questions?
\end{frame}

\end{document}
